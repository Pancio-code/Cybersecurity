\UseRawInputEncoding
\documentclass{article}

% Language setting
\usepackage[english]{babel}
\usepackage{pmboxdraw}

\usepackage[a4paper,top=2cm,bottom=2cm,left=3cm,right=3cm,marginparwidth=1.75cm]{geometry}

% Useful packages
\usepackage{amsmath}
\usepackage{tabularx}
\usepackage[T1]{fontenc}
\usepackage{graphicx}
\usepackage{listings}
\usepackage[colorlinks=true, allcolors=blue]{hyperref}

\title{Homework 10\\"Offline dictionary attack"}

\author{Andrea Panceri 1884749}

\begin{document}
\maketitle

\section{Introduction}
An offline dictionary attack is a type of password cracking method in which an attacker pre-computes the hashes of a large number of common words and phrases, and then compares these hashes to the hashed password they are trying to crack. The goal of this attack is to find a match between the pre-computed hashes and the target hashed password, which would allow the attacker to determine the original password.\\
This type of attack is called "offline" because the attacker does not need to interact with the system that is being attacked in order to perform the attack. Instead, the attacker can simply run the attack on a separate device, using a pre-computed dictionary of hashes.\\
One of the main advantages of an offline dictionary attack is that it can be very fast, as the attacker does not need to wait for a response from the target system for each attempted password. However, this type of attack is also limited in its effectiveness, as it relies on the assumption that the target password is a common word or phrase that is included in the attacker's pre-computed dictionary. If the target password is not a common word or phrase, or if it has been modified in some way (e.g., by adding numbers or special characters), then the attack is unlikely to be successful.\\
To protect against offline dictionary attacks, it is important to use strong, unique passwords that are not based on common words or phrases, and to use password hashing algorithms that are resistant to pre-computation attacks. It is also a good idea to use multi-factor authentication, which requires the user to provide additional evidence of their identity in order to gain access to a system.\\
\section{Script}
We have this file with username:salt:hashed\_password, the hashing function used is sha256:
\begin{lstlisting}
user01:30:e9b5067a9ac080a346df5ff6fdda9c0ee015ca85f1e689c0eb129e8a3799e1a7
user02:22:78b868e79d1efeb8dd4fe475696667b0710607dc96372173ee62d01ed2f721d1
user03:81:cd778a65aff718dc3dc2fa5baa8afd7086e9d9569d5378c4e873febb2459b86c
user04:67:39e922f77cf18c4807e73624a518526e0a02fe7e51f8c90a63b811a647aff621
user05:47:397f5af8ba556c09ad2ad96d5c2c5f555ffcbc36e1722a68d24db7ecb6baf92c
user06:10:2034444c0692c122fef1edbed3ddbf110c1623b00afeb009741076dde4288368
user07:45:2de7f0807464d3fd0a7b6209168cc3387b0d61e3048a18b673a89b0bfca04968
user08:96:cf7152cf48e93ff0fb9adc50c74dfe4cd98fff15cba23de0467e610e356a6d2b
user09:49:6e5bb8d71d7320ea65a032d42d56a5e431563231ac0bde932a7c5359d83eace4
user10:9:65f1d91cb5d4745410f91aec14209bb8f17537f401a6d298dd9d76425950f616
\end{lstlisting}
Now we report two different approach with python
\begin{lstlisting}
from urllib.request import urlopen, hashlib

# Read the file containing the username, salt, and hashed password
with open("passwd_file", "r") as f:
    lines = f.readlines()

# Load the dictionary of common words and phrases
with open("dictionary.txt", "r") as f:
    words = f.read().splitlines()

# Iterate over each user in the password file
for user in lines:
    # Split the line into the username, salt, and hashed password
    username, salt, hashed_password = user.strip().split(":")

    # Iterate over each word in the dictionary
    for word in words:
        # Hash the word with the salt and compare it to the hashed password
        word_hash = hashlib.sha256((word.strip() + salt).encode()).hexdigest()
        if word_hash == hashed_password:
            print(f"Found password for user {username}: {word}")
            break
\end{lstlisting}
In this simple script we load the password file given by email, and a local dictionary downloaded from internet (there are many available. For each user we try to hash with sha256 each word in dictionary appending the salt after the word, and compare with the hashed password in the file. Another approach is this in which we don't use a local file but load a dictionary from an online github repository:
\begin{lstlisting}
from urllib.request import urlopen, hashlib

# Read the file containing the username, salt, and hashed password
with open("passwd_file", "r") as f:
    lines = f.readlines()

# Load the dictionary of common words and phrases
words = str(urlopen(
    'https://raw.githubusercontent.com/Legoclones/password-cracking/
        main/dictionaries/engDictionaryLarge.txt').read(), 'utf-8')

# Iterate over each user in the password file
for user in lines:
    # Split the line into the username, salt, and hashed password
    username, salt, hashed_password = user.strip().split(":")

    # Iterate over each word in the dictionary
    for word in words.split('\n'):
        # Hash the word with the salt and compare it to the hashed password
        word_hash = hashlib.sha256((word.strip() + salt).encode()).hexdigest()
        if word_hash == hashed_password:
            print(f"Found password for user {username}: {word}")
            break

\end{lstlisting}
I try the first method, i download many dictionary with millions of leaked password (like rockyou leak) but i don't obtain any match. Also the second method i try to use very large file with all combination of words and symnols and don't obtain nothing.\\
I try another approach but it is needed i very high computational power, i report the code:
\begin{lstlisting}[basicstyle=\tiny]
from urllib.request import urlopen, hashlib
import string

characters = list(string.printable)  # all typeable characters

# Read the file containing the username, salt, and hashed password
with open("passwd_file", "r") as f:
    lines = f.readlines()

for first in characters:
    for user in lines:
        # Split the line into the username, salt, and hashed password
        username, salt, hashed_password = user.strip().split(":")
        word = first
        # Hash the word with the salt and compare it to the hashed password
        word_hash = hashlib.sha256((word.strip() + salt).encode()).hexdigest()
        if word_hash == hashed_password:
            print(f"Found password for user {username}: {word}")
            break
    for second in characters:
        for user in lines:
            # Split the line into the username, salt, and hashed password
            username, salt, hashed_password = user.strip().split(":")
            word = first + second
            # Hash the word with the salt and compare it to the hashed password
            word_hash = hashlib.sha256(
                (word.strip() + salt).encode()).hexdigest()
            if word_hash == hashed_password:
                print(f"Found password for user {username}: {word}")
                break
        for third in characters:
            for user in lines:
                # Split the line into the username, salt, and hashed password
                username, salt, hashed_password = user.strip().split(":")
                word = first + second + third
                # Hash the word with the salt and compare it to the hashed password
                word_hash = hashlib.sha256(
                    (word.strip() + salt).encode()).hexdigest()
                if word_hash == hashed_password:
                    print(f"Found password for user {username}: {word}")
                    break
            for fourth in characters:
                for user in lines:
                    # Split the line into the username, salt, and hashed password
                    username, salt, hashed_password = user.strip().split(":")
                    word = first + second + third + fourth
                    # Hash the word with the salt and compare it to the hashed password
                    word_hash = hashlib.sha256(
                        (word.strip() + salt).encode()).hexdigest()
                    if word_hash == hashed_password:
                        print(f"Found password for user {username}: {word}")
                        break
                for fifth in characters:
                    for user in lines:
                        # Split the line into the username, salt, and hashed password
                        username, salt, hashed_password = user.strip().split(":")
                        word = first + second + third + fourth + fifth
                        # Hash the word with the salt and compare it to the hashed password
                        word_hash = hashlib.sha256(
                            (word.strip() + salt).encode()).hexdigest()
                        if word_hash == hashed_password:
                            print(f"Found password for user {username}: {word}")
                            break
                    for sixth in characters:
                        for user in lines:
                            # Split the line into the username, salt, and hashed password
                            username, salt, hashed_password = user.strip().split(":")
                            word = first + second + third + fourth + fifth + sixth
                            # Hash the word with the salt and compare it to the hashed password
                            word_hash = hashlib.sha256(
                                (word.strip() + salt).encode()).hexdigest()
                            if word_hash == hashed_password:
                                print(f"Found password for user {username}: {word}")
                                break
                        for seventh in characters:
                            for user in lines:
                                # Split the line into the username, salt, and hashed password
                                username, salt, hashed_password = user.strip().split(":")
                                word = first + second + third + fourth + fifth + sixth + seventh
                                # Hash the word with the salt and compare it to the hashed password
                                word_hash = hashlib.sha256(
                                    (word.strip() + salt).encode()).hexdigest()
                                if word_hash == hashed_password:
                                    print(f"Found password for user {username}: {word}")
                                    break
                            for eigth in characters:
                                print(word)
                                for user in lines:
                                    # Split the line into the username, salt, and hashed password
                                    username, salt, hashed_password = user.strip().split(":")
                                    word = first + second + third + fourth + fifth + sixth + seventh + eigth
                                    # Hash the word with the salt and compare it to the hashed password
                                    word_hash = hashlib.sha256(
                                        (word.strip() + salt).encode()).hexdigest()
                                    if word_hash == hashed_password:
                                        print(f"Found password for user {username}: {word}")
                                        break
\end{lstlisting}
In this script we try all possible combination of type able string of length 1 to 8 for each user, but this require an incredible amount of combination, the type able characters are 128 and we do 128 power of 8 tries for each user (ten users), the script is running from three days, but until now doesn't find any matching password. This script doesn't guarantee to find the password, because the password could be longer than eighth characters. \\
In summary these scripts all works but i doesn't succeed to find any of the passwords, it is simply needed a dictionary with the correct passwords, but i didn't find it. With the right dictionary and a bit of luck it is very fast to find the match.\\
In the email there are all files used!

\end{document}
